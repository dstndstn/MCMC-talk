\pdfobjcompresslevel=1
\documentclass{beamer}
\usepackage{pdfpages}
\usepackage{mathtools}
%\usepackage{amsmath}
\usepackage{tikz}
%\usetikzlibrary{arrows,decorations.pathmorphing,backgrounds,placments,fit}
\usetikzlibrary{arrows.meta,decorations.pathmorphing,backgrounds,positioning,fit}

\usepackage{minted}

%\usepackage{enumitem}

%\usepackage[wby]{callouts}

\newcommand{\dfmpage}[1]{
{
\setbeamercolor{background canvas}{bg=}
\includepdf[pages=#1]{dfm.pdf}
}
}

% DFM made a typo in the key Metropolis-Hastings acceptance probability, so patch those slides with an overlay!!

\newcommand{\fixequation}{
\AddToHookNext{shipout/foreground}{
  \begin{tikzpicture}[overlay, remember picture]
    %\draw[step=1cm] (0, 0) grid (10, -5);
    \node[draw=none, fill=white, text=black, scale=1] at (6.7, -4) {%
      $p( \textrm{accept} ) = \min \left( 1, \displaystyle\frac{p(\mathbf{x}')}{p(\mathbf{x})} \frac{q(\mathbf{x}; \mathbf{x}')}{q(\mathbf{x}'; \mathbf{x})} \right)$\hspace{1em}};
  \end{tikzpicture}
}
}

\newcommand{\fixequationx}{
\AddToHookNext{shipout/foreground}{
  \begin{tikzpicture}[overlay, remember picture]
    \node[draw=none, fill=white, text=black, scale=1] at (6.7, -5.2) {%
      $p( \textrm{accept} ) = \min \left( 1, \displaystyle\frac{p(\mathbf{x}')}{p(\mathbf{x})} \frac{q(\mathbf{x}; \mathbf{x}')}{q(\mathbf{x}'; \mathbf{x})} \right)$\hspace{1em}};
  \end{tikzpicture}
}
}

\def\Put(#1,#2)#3{\leavevmode\makebox(0,0){\put(#1,#2){#3}}}

% fonts p14; 18.2.3
% Futura font

% Cambridge, Copenhagen, JuanLesPins, Luebeck, Malmoe, Marburg,
% Montpellier, PaloAlto, Singapore

% colortheme beaver, dolphin

% Copyright 2004 by Till Tantau <tantau@users.sourceforge.net>.
%
% In principle, this file can be redistributed and/or modified under
% the terms of the GNU Public License, version 2.
%
% However, this file is supposed to be a template to be modified
% for your own needs. For this reason, if you use this file as a
% template and not specifically distribute it as part of a another
% package/program, I grant the extra permission to freely copy and
% modify this file as you see fit and even to delete this copyright
% notice. 

\mode<presentation> {
  \usetheme{Malmoe}
  \usecolortheme{beaver}
  \setbeamercovered{transparent}
  \setbeamertemplate{navigation symbols}{{\small\insertpagenumber}}
%{{\normalsize\insertframenumber}}
  \setbeamertemplate{footline}{%
    \leavevmode%
    \hbox{\begin{beamercolorbox}[wd=\paperwidth,ht=0.5ex,dp=1.125ex,leftskip=.3cm,rightskip=.3cm plus1fil]{title in head/foot}%
    \end{beamercolorbox}}%
    \vskip0pt%
  }
  \setbeamertemplate{headline}{% %split theme}
  \leavevmode%
    \begin{beamercolorbox}[wd=.3\paperwidth,ht=2.5ex,dp=1.125ex]{section in head/foot}%
      \insertsectionnavigationhorizontal{.3\paperwidth}{\hskip0pt plus1filll}{}%
  \end{beamercolorbox}%
  \begin{beamercolorbox}[wd=.7\paperwidth,ht=2.5ex,dp=1.125ex]{subsection in head/foot}%
    \insertsubsectionnavigationhorizontal{.7\paperwidth}{}{\hskip0pt plus1filll}%
  \end{beamercolorbox}%
  }
  %\setbeamersize{sidebar width right=2ex}
  %{\usebeamercolor{sidebar}}
  %\setbeamertemplate{sidebar canvas right}{f \insertframenumber}
  %\insertpagenumber
}

\usepackage[english]{babel}
\usepackage[latin1]{inputenc}
\usepackage{helvet}
\usepackage{xspace}
% Or whatever. Note that the encoding and the font should match. If T1
% does not look nice, try deleting the line with the fontenc.
%\usepackage[T1]{fontenc}
\usepackage[normalem]{ulem}
\usepackage{calc}
\usepackage{verbatim}
\usepackage{multirow}
\usepackage{dcolumn}
\usepackage{multimedia} 
%\usepackage{amsbsy}
\usepackage{amsmath}

\newcommand{\arxiv}[1]{\href{http://arxiv.org/abs/#1}{arXiv:#1}}
\newcommand{\etal}{\textit{et al.~}}
\newcommand{\snr}[1]{\mathbb{SN}(#1)}


\graphicspath{{figs-slides/}{figs-techreport/}}

\newcommand{\an}{\emph{Astrometry.net}\xspace}
\newcommand{\libkd}{\emph{libkd}\xspace}
\newcommand{\kdtree}{$kd$-tree}
\newcommand{\antoc}{Astrometry.net\xspace}
\newcommand{\eg}{\emph{eg}}

% holmes
\newcommand{\light}[1]{{\color{gray}#1}}

\newcommand{\paramvector}[1]{\boldsymbol{#1}}
\newcommand{\pointing}{\paramvector{\alpha}}
\newcommand{\fovpars}{\paramvector{\Omega}}
\newcommand{\orbitpars}{\paramvector{\omega}}
\newcommand{\hyperpars}{\paramvector{\theta}}
\newcommand{\position}{\paramvector{x}}
\newcommand{\velocity}{\paramvector{v}}
\newcommand{\uniform}{\mathrm{uniform}}
\newcommand{\tmin}{t_\mathrm{min}}
\newcommand{\tmax}{t_\mathrm{max}}
\newcommand{\pgood}{p_\mathrm{good}}
\newcommand{\pempirical}{p_\mathrm{emp}}
\newcommand{\pemp}{\pempirical}
\newcommand{\exif}{\mathrm{EXIF}}
\newcommand{\pexif}{p_\exif}
\newcommand{\texif}{t_\exif}
\newcommand{\pfg}{p_\mathrm{fg}}
\newcommand{\pbg}{p_\mathrm{bg}}

% commands to add more space in \itemize environments
\newcommand{\bitmorespace}{%
  \addtolength{\itemsep}{0.5ex}%
  %\addtolength{\parskip}{0.5ex}%
  %\addtolength{\parsep}{0.5ex}%
  %\addtolength{\topsep}{0.5ex}%
  \vspace{0.5ex}%
}
\newcommand{\morespace}{\addtolength{\itemsep}{1ex}}
\newcommand{\Morespace}{\addtolength{\itemsep}{1.5ex}}


\newcommand{\commentout}[1]{}


\usefonttheme[onlymath]{serif}
\usepackage{multimedia} 

\newcommand{\niceurl}[1]{\mbox{\href{#1}{\textsl{#1}}}}

\title{Monte Carlo Methods}
\author{Dustin Lang \\
Perimeter Institute for Theoretical Physics}
\date{PSI Numerical Methods 2026 \\
  \vspace{1em}
These slides are available at \niceurl{https://github.com/dstndstn/MCMC-talk}%
}

%\setlist[itemize]{leftmargin=*}

\begin{document}

\begin{frame}
\titlepage
\end{frame}

\begin{frame}{Monte Carlo}
\begin{itemize}
\item A set of computational approaches that \\
uses \emph{repeated random simulations}
\item Often: \emph{numerical integration}, \\
\emph{optimization}, and \emph{sampling from \\
probability distributions}
\item \emph{Monte Carlo} is a casino in Monaco -- a reference to the \emph{randomness} used in
these algorithms
\item Developed at the very beginning of computing: 1940s, Los Alamos, nuclear bomb work
(diffusion of neutrons in fissionable material)
%\Put(10,50){\includegraphics[width=0.3\textwidth]{demoncore}}
\item Compute \emph{Expectation values} over probability distributions:
\item $\mathbb{E}(\phi(X)) = \int p(X) \phi(X) dX \sim \frac{1}{N} \sum_i^N \phi(X_i) $
\end{itemize}

\begin{tikzpicture}[remember picture,overlay]
    \node[xshift=-2.1cm,yshift=-3cm] at (current page.north east){%
    \includegraphics[width=4cm]{demoncore}};
\end{tikzpicture}
\end{frame}

\begin{frame}{Monte Carlo: Examples}
From \niceurl{https://en.wikipedia.org/wiki/Monte\_Carlo\_method}

\begin{columns}
\begin{column}{0.4\textwidth}
\includegraphics[width=\textwidth]{Pi_monte_carlo_all-6}
\end{column}
\begin{column}{0.6\textwidth}
    \setlength{\itemindent}{-2em}
    \begin{itemize}%[leftmargin=*]
    \item Monte Carlo estimator: $\hat{\theta} = \frac{1}{N} \sum_{i=1}^N \phi(x_i)$
    \item Variance: $\textrm{var}(\hat{\theta}) = \frac{1}{N} \textrm{var}(\phi(X))$ \\
    (of the \emph{distribution} $X$)
    \item Sample v:  $\textrm{var}(\hat{\theta}) \simeq \frac{1}{N} \left( \frac{1}{N-1} \sum (\phi(x_i) - \hat{\theta})^2 \right)$
    \item {\footnotesize see \niceurl{https://mpaldridge.github.io/math5835/lectures/L03-mc-error-1.html}
    for proofs}
    \end{itemize}
\end{column}
\end{columns}

\end{frame}

\begin{frame}{Monte Carlo examples: Path Tracing}
\includegraphics[height=0.6\textwidth]{path-tracing} \\
{\footnotesize Credit: Wikimedia, \niceurl{https://commons.wikimedia.org/wiki/File:Path\_tracing\_001.png}}
\end{frame}

\begin{frame}{Monte Carlo examples: Path Tracing}
\includegraphics[height=0.6\textwidth]{Path-tracing-samples} \\
{\footnotesize Credit: Wikimedia, \niceurl{https://en.wikipedia.org/wiki/Path\_tracing\#/media/File:Path\_tracing\_sampling\_values.png}}
\end{frame}

\begin{frame}{Monte Carlo examples: Quantum Monte Carlo}
\begin{itemize}
\item Dealing with multi-particle wavefunctions often involves many-dimensional integrals
\item \emph{The Curse of Dimensionality}
\item Variational Monte Carlo: uses Monte Carlo sampling of electron positions in real space
to estimate the ground-state wavefunction, given a Hamiltonian
\item Parameterized wavefunction $|\psi(a)\rangle$
\item Estimate $E(a) = \frac{\langle \psi(a) | H | \psi(a) \rangle}{\langle \psi(a) | \psi(a) \rangle}$
\item ie, the integral $E(a) = \frac{\int | \psi(X, a) |^2 \frac{H \psi(X, a)}{\psi(X,a)} dX}%
      {\int | \psi(X, a) |^2 dX}$
\end{itemize}
\begin{tikzpicture}[remember picture,overlay]
    \node[xshift=-1.1cm,yshift=2cm] at (current page.south east){%
    \includegraphics[width=0.2\textwidth]{benzene}\\
    {\tiny From \niceurl{https://doi.org/10.1134/S2070048220020106}}};
\end{tikzpicture}
%\includegraphics[width=0.2\textwidth]{benzene}
%{\tiny From \niceurl{https://doi.org/10.1134/S2070048220020106}}
\end{frame}

\begin{frame}{Monte Carlo examples: Quantum Monte Carlo}
\begin{itemize}
\item ie, the integral $E(a) = \frac{\int | \psi(X, a) |^2 \frac{H \psi(X, a)}{\psi(X,a)} dX}%
      {\int | \psi(X, a) |^2 dX}$
\item Treat this term as a probability distribution \\
function: $p(X) = \frac{| \psi(X, a) |^2}{\int | \psi(X, a) |^2 dX}$
\item Then we've got $E(a) = \int p(X) \frac{H \psi(X,a)}{\psi(X,a)}  dX$
\item Monte Carlo it!
\item $E(a) \simeq \frac{1}{N} \sum \frac{H \psi(X_i, a)}{\psi(X_i, a)}$
\item with $X_i$ drawn from the \\
``trial wavefunction'' $\psi(a)$
\end{itemize}
\begin{tikzpicture}[remember picture,overlay]
    \node[xshift=-1.1cm,yshift=2cm] at (current page.south east){%
    \includegraphics[width=0.2\textwidth]{benzene}\\
    {\tiny From \niceurl{https://doi.org/10.1134/S2070048220020106}}};
\end{tikzpicture}
%\includegraphics[width=0.2\textwidth]{benzene}
%{\tiny From \niceurl{https://doi.org/10.1134/S2070048220020106}}
\end{frame}


\end{document}

