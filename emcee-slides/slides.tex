\pdfobjcompresslevel=1
\documentclass{beamer}
\usepackage{pdfpages}
\usepackage{mathtools}
%\usepackage{amsmath}
\usepackage{tikz}
%\usetikzlibrary{arrows,decorations.pathmorphing,backgrounds,placments,fit}
\usetikzlibrary{arrows.meta,decorations.pathmorphing,backgrounds,positioning,fit}

\usepackage{minted}
\usepackage{animate}
%\usepackage{movie15}
\usepackage[export]{adjustbox}

\usepackage{xmpmulti}

\newcommand{\dfmpage}[1]{
{
\setbeamercolor{background canvas}{bg=}
\includepdf[pages=#1]{dfm.pdf}
}
}

\newcommand{\jrppage}[1]{
{
\setbeamercolor{background canvas}{bg=}
\includepdf[pages=#1]{jrp.pdf}
}
}

% fonts p14; 18.2.3
% Futura font

% Cambridge, Copenhagen, JuanLesPins, Luebeck, Malmoe, Marburg,
% Montpellier, PaloAlto, Singapore

% colortheme beaver, dolphin

% Copyright 2004 by Till Tantau <tantau@users.sourceforge.net>.
%
% In principle, this file can be redistributed and/or modified under
% the terms of the GNU Public License, version 2.
%
% However, this file is supposed to be a template to be modified
% for your own needs. For this reason, if you use this file as a
% template and not specifically distribute it as part of a another
% package/program, I grant the extra permission to freely copy and
% modify this file as you see fit and even to delete this copyright
% notice. 

\mode<presentation> {
  \usetheme{Malmoe}
  \usecolortheme{beaver}
  \setbeamercovered{transparent}
  \setbeamertemplate{navigation symbols}{{\small\insertpagenumber}}
%{{\normalsize\insertframenumber}}
  \setbeamertemplate{footline}{%
    \leavevmode%
    \hbox{\begin{beamercolorbox}[wd=\paperwidth,ht=0.5ex,dp=1.125ex,leftskip=.3cm,rightskip=.3cm plus1fil]{title in head/foot}%
    \end{beamercolorbox}}%
    \vskip0pt%
  }
  \setbeamertemplate{headline}{% %split theme}
  \leavevmode%
    \begin{beamercolorbox}[wd=.3\paperwidth,ht=2.5ex,dp=1.125ex]{section in head/foot}%
      \insertsectionnavigationhorizontal{.3\paperwidth}{\hskip0pt plus1filll}{}%
  \end{beamercolorbox}%
  \begin{beamercolorbox}[wd=.7\paperwidth,ht=2.5ex,dp=1.125ex]{subsection in head/foot}%
    \insertsubsectionnavigationhorizontal{.7\paperwidth}{}{\hskip0pt plus1filll}%
  \end{beamercolorbox}%
  }
  %\setbeamersize{sidebar width right=2ex}
  %{\usebeamercolor{sidebar}}
  %\setbeamertemplate{sidebar canvas right}{f \insertframenumber}
  %\insertpagenumber
}

\usepackage[english]{babel}
\usepackage[latin1]{inputenc}
\usepackage{helvet}
\usepackage{xspace}
% Or whatever. Note that the encoding and the font should match. If T1
% does not look nice, try deleting the line with the fontenc.
%\usepackage[T1]{fontenc}
\usepackage[normalem]{ulem}
\usepackage{calc}
\usepackage{verbatim}
\usepackage{multirow}
\usepackage{dcolumn}
\usepackage{multimedia} 
%\usepackage{amsbsy}
\usepackage{amsmath}

\newcommand{\arxiv}[1]{\href{http://arxiv.org/abs/#1}{arXiv:#1}}
\newcommand{\etal}{\textit{et al.~}}
\newcommand{\snr}[1]{\mathbb{SN}(#1)}


\graphicspath{{figs-slides/}{figs-techreport/}}

\newcommand{\an}{\emph{Astrometry.net}\xspace}
\newcommand{\libkd}{\emph{libkd}\xspace}
\newcommand{\kdtree}{$kd$-tree}
\newcommand{\antoc}{Astrometry.net\xspace}
\newcommand{\eg}{\emph{eg}}

% holmes
\newcommand{\light}[1]{{\color{gray}#1}}

\newcommand{\paramvector}[1]{\boldsymbol{#1}}
\newcommand{\pointing}{\paramvector{\alpha}}
\newcommand{\fovpars}{\paramvector{\Omega}}
\newcommand{\orbitpars}{\paramvector{\omega}}
\newcommand{\hyperpars}{\paramvector{\theta}}
\newcommand{\position}{\paramvector{x}}
\newcommand{\velocity}{\paramvector{v}}
\newcommand{\uniform}{\mathrm{uniform}}
\newcommand{\tmin}{t_\mathrm{min}}
\newcommand{\tmax}{t_\mathrm{max}}
\newcommand{\pgood}{p_\mathrm{good}}
\newcommand{\pempirical}{p_\mathrm{emp}}
\newcommand{\pemp}{\pempirical}
\newcommand{\exif}{\mathrm{EXIF}}
\newcommand{\pexif}{p_\exif}
\newcommand{\texif}{t_\exif}
\newcommand{\pfg}{p_\mathrm{fg}}
\newcommand{\pbg}{p_\mathrm{bg}}

% commands to add more space in \itemize environments
\newcommand{\bitmorespace}{%
  \addtolength{\itemsep}{0.5ex}%
  %\addtolength{\parskip}{0.5ex}%
  %\addtolength{\parsep}{0.5ex}%
  %\addtolength{\topsep}{0.5ex}%
  \vspace{0.5ex}%
}
\newcommand{\morespace}{\addtolength{\itemsep}{1ex}}
\newcommand{\Morespace}{\addtolength{\itemsep}{1.5ex}}


\newcommand{\commentout}[1]{}


\usefonttheme[onlymath]{serif}
\usepackage{multimedia} 

\newcommand{\niceurl}[1]{\mbox{\href{#1}{\textsl{#1}}}}

\title{emcee: An Affine-Invariant Sampler}
\author{Dustin Lang \\
Perimeter Institute for Theoretical Physics}
\date{Symmetries Graduate School 2023-01-30 \\
  \vspace{1em}
Borrowing heavily from Dan Foreman-Mackey's slides \niceurl{https://speakerdeck.com/dfm/data-analysis-with-mcmc1}
  \vspace{1em}
These slides are available at \niceurl{https://github.com/dstndstn/MCMC-talk/emcee-slides}%
}
\begin{document}

\begin{frame}
\titlepage
\end{frame}

\begin{frame}{Recap from last week's lecture (1)}
\begin{itemize}
\addtolength{\itemsep}{0.5em}
\item Markov Chain Monte Carlo (MCMC) 
\emph{draws samples from a probability distribution}
when you can \emph{numerically evaluate} the probability function
(up to a constant)
\item Used extensively in data analysis:
\emph{inferring} parameters of models, given observed data
\item \emph{Usually} in a Bayesian context; the probability function we
run MCMC on is the \emph{posterior} probability: \\
$\textrm{posterior}(\textrm{params} | \textrm{data}) \propto$ \\
$\quad \textrm{prior}(\textrm{params}) \times \textrm{likelihood}(\textrm{data} | \textrm{params})$
\end{itemize}

\vspace{-0.5em}
\centering
\includegraphics[valign=t,width=0.4\textwidth]{pm2}
\hspace{0.1\textwidth}
\includegraphics[valign=t,width=0.2\textwidth]{pm-constraints}
\end{frame}
 

\begin{frame}{Recap from last week's lecture (2)}
\begin{itemize}
\addtolength{\itemsep}{1em}
\item The ``classic'' Markov Chain Monte Carlo algorithm is
\emph{Metropolis--Hastings}, which moves a \emph{walker} or \emph{particle}
around the \emph{state space} (\emph{model parameter space})
\item A randomly-drawn \emph{proposed} jump gets \emph{evaluated} (by calling
the probability function), and then \emph{accepted}, or not
\item A big difficulty is to \emph{customize} the \emph{proposal distribution}
to get the algorithm to work efficiently
\end{itemize}

\vspace{-0.5em}
\centering
\includegraphics[valign=t,page=44,width=0.4\textwidth]{dfm}
\end{frame}



% \begin{frame}[fragile]{The MCMC Algorithm}
% \begin{scriptsize}
% \begin{minted}{python}
% def mcmc(logprob_func, propose_func, initial_pos, nsteps):
%      p = initial_pos
%      logprob = logprob_func(p)
%      chain = []
%      for i in range(nsteps):
%          # propose a new position in parameter space
%          p_new = propose_func(p)
%          # compute probability at new position
%          logprob_new = logprob_func(p_new)
%          # decide whether to jump to the new position
%          if exp(logprob_new - logprob) > uniform_random():
%              p = p_new
%              logprob = logprob_new
%          # save the position
%          chain.append(p)
%      return chain
% \end{minted}
% \end{scriptsize}
% \end{frame}

\begin{frame}{MCMC for model parameter inference}
  \includegraphics[height=0.8\textheight]{mcmc-results}
\end{frame}


\dfmpage{50}
\dfmpage{53}
\dfmpage{61-62}
\dfmpage{64}
\dfmpage{6}

%\dfmpage{70-75}
\dfmpage{70-74}
\dfmpage{75}

% \begin{frame}
% \begin{itemize}
% \item Draw a uniform random number $u$
% \item ``Stretch factor'' $z = \frac{(u + 1)^2}{2}$
% \end{itemize}
% \end{frame}


%\dfmpage{79}

\begin{frame}{Emcee demo}
%\centering
%\scalebox{0.5}{
%}
% DOESNOT WORK
%\multiinclude[<+>][format=png,start=0,end=9,graphics={height=0.8\textheight}]{emcee/emcee}
%\multiinclude[<+->][format=png,start=0,end=9,graphics={height=0.8\textheight}]{emcee/emcee}
\begin{center}
\only<1>{\includegraphics[height=0.8\textheight]{emcee/emcee-0.png}}%
\only<2>{\includegraphics[height=0.8\textheight]{emcee/emcee-1.png}}%
\only<3>{\includegraphics[height=0.8\textheight]{emcee/emcee-2.png}}%
\only<4>{\includegraphics[height=0.8\textheight]{emcee/emcee-3.png}}%
\only<5>{\includegraphics[height=0.8\textheight]{emcee/emcee-4.png}}%
\only<6>{\includegraphics[height=0.8\textheight]{emcee/emcee-5.png}}%
\only<7>{\includegraphics[height=0.8\textheight]{emcee/emcee-6.png}}%
\only<8>{\includegraphics[height=0.8\textheight]{emcee/emcee-7.png}}%
\only<9>{\includegraphics[height=0.8\textheight]{emcee/emcee-8.png}}%
\only<10>{\includegraphics[height=0.8\textheight]{emcee/emcee-9.png}}%
\only<11>{\includegraphics[height=0.8\textheight]{emcee/emcee-10.png}}%
\only<12>{\includegraphics[height=0.8\textheight]{emcee/emcee-11.png}}%
\only<13>{\includegraphics[height=0.8\textheight]{emcee/emcee-12.png}}%
\only<14>{\includegraphics[height=0.8\textheight]{emcee/emcee-13.png}}%
\only<15>{\includegraphics[height=0.8\textheight]{emcee/emcee-14.png}}%
\only<16>{\includegraphics[height=0.8\textheight]{emcee/emcee-15.png}}%
\only<17>{\includegraphics[height=0.8\textheight]{emcee/emcee-16.png}}%
\only<18>{\includegraphics[height=0.8\textheight]{emcee/emcee-17.png}}%
\only<19>{\includegraphics[height=0.8\textheight]{emcee/emcee-18.png}}%
\only<20>{\includegraphics[height=0.8\textheight]{emcee/emcee-19.png}}%
\only<21>{\includegraphics[height=0.8\textheight]{emcee/emcee-20.png}}%
\only<22>{\includegraphics[height=0.8\textheight]{emcee/emcee-21.png}}%
\only<23>{\includegraphics[height=0.8\textheight]{emcee/emcee-22.png}}%
\only<24>{\includegraphics[height=0.8\textheight]{emcee/emcee-23.png}}%
\only<25>{\includegraphics[height=0.8\textheight]{emcee/emcee-24.png}}%
\only<26>{\includegraphics[height=0.8\textheight]{emcee/emcee-25.png}}%
\only<27>{\includegraphics[height=0.8\textheight]{emcee/emcee-26.png}}%
\only<28>{\includegraphics[height=0.8\textheight]{emcee/emcee-27.png}}%
\only<29>{\includegraphics[height=0.8\textheight]{emcee/emcee-28.png}}%
\only<30>{\includegraphics[height=0.8\textheight]{emcee/emcee-29.png}}%
\only<31>{\includegraphics[height=0.8\textheight]{emcee/emcee-30.png}}%
\only<32>{\includegraphics[height=0.8\textheight]{emcee/emcee-31.png}}%
\only<33>{\includegraphics[height=0.8\textheight]{emcee/emcee-32.png}}%
\only<34>{\includegraphics[height=0.8\textheight]{emcee/emcee-33.png}}%
\only<35>{\includegraphics[height=0.8\textheight]{emcee/emcee-34.png}}%
\only<36>{\includegraphics[height=0.8\textheight]{emcee/emcee-35.png}}%
\only<37>{\includegraphics[height=0.8\textheight]{emcee/emcee-36.png}}%
\only<38>{\includegraphics[height=0.8\textheight]{emcee/emcee-37.png}}%
\only<39>{\includegraphics[height=0.8\textheight]{emcee/emcee-38.png}}%
\only<40>{\includegraphics[height=0.8\textheight]{emcee/emcee-39.png}}%
\end{center}
%\multiinclude[format=png,start=0,end=39,height=0.8\textheight]{emcee/emcee}
%\multiinclude[format=png,start=0,end=9,height=0.8\textheight]{emcee/emcee-B}
%\includegraphics[height=0.8\textheight]{emcee/emcee-00}
\end{frame}

\begin{frame}{Emcee demo}
\begin{center}
\only<1>{\includegraphics[height=0.8\textheight]{emcee/emcee-B-0.png}}%
\only<2>{\includegraphics[height=0.8\textheight]{emcee/emcee-B-1.png}}%
\only<3>{\includegraphics[height=0.8\textheight]{emcee/emcee-B-2.png}}%
\only<4>{\includegraphics[height=0.8\textheight]{emcee/emcee-B-3.png}}%
\only<5>{\includegraphics[height=0.8\textheight]{emcee/emcee-B-4.png}}%
\only<6>{\includegraphics[height=0.8\textheight]{emcee/emcee-B-5.png}}%
\only<7>{\includegraphics[height=0.8\textheight]{emcee/emcee-B-6.png}}%
\only<8>{\includegraphics[height=0.8\textheight]{emcee/emcee-B-7.png}}%
\only<9>{\includegraphics[height=0.8\textheight]{emcee/emcee-B-8.png}}%
\only<10>{\includegraphics[height=0.8\textheight]{emcee/emcee-B-9.png}}%
\end{center}
\end{frame}

%\dfmpage{81}
%\dfmpage{106-111}
\dfmpage{107-109}

\begin{frame}{Differential Evolution move}
\begin{itemize}
\item \alert{emcee} allows us to use different \emph{move} types
(different \emph{proposal} functions)
\item The \alert{Differential Evolution} (DE) move can improve the sampling
for multi-modal distributions
\item DE move: randomly select \emph{two} ``helpers''
\item Propose moving by their \alert{vector difference}
\item (If they are from different modes, this proposes \emph{jumping between modes})
\item Mixing in a fraction of DE moves with the regular ``Stretch'' move works well!
\end{itemize}
\end{frame}

\dfmpage{109}

%%%%%%%%%%%%%%%%%%%%%%%%%%%%%%%%%%%%%%


\begin{frame}{Summary}
\begin{itemize}
\addtolength{\itemsep}{0.5em}
\item Traditional Metropolis--Hastings MCMC suffers from a \emph{lack of affine invariance} -- requires \emph{tuning parameters} that change for each specific probability function
\item \emph{Ensemble samplers} like \alert{emcee} use the
\emph{distribution of the walkers} to achieve \emph{affine invariance}
\item $\to$ much easier to use, and faster sampling
\item (Huge side effect: parallelizable!)
\item Multi-modal distributions still hard, but \emph{DE Move} can help
\item MCMC isn't scary!
\end{itemize}
\end{frame}

\end{document}

